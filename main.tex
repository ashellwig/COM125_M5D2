\documentclass[stu,12pt]{apa7}
  \usepackage{times}               % Times New Roman Font Face
  \usepackage[american]{babel}     % Localization
  \usepackage[utf8]{inputenc}      % Input Encoding
  \usepackage{hyperref}            % Hyperlinks
  \usepackage{enumitem}            % Additional Enumeration Environment Settings
  \usepackage{geometry}            % Page Layout
  \usepackage{soul}                % Text Highlighting
  \usepackage{graphicx}            % Images
  \usepackage{csquotes}            % Quoting Environment
  \usepackage{bookmark}            % Required by `csquotes'
  \usepackage{mdframed}            % Colorful Tex-Box Environment
  \usepackage[toc]{appendix}       % Appendix
  \usepackage{fancyhdr}            % Headings and Footers
  \usepackage{xcolor}              % Text Colors
  \usepackage[style=apa,sortcites=true,sorting=nyt]{biblatex}  % Citations


  % Bibliography Setup
  %% Language Mappings
  \DeclareLanguageMapping{english}{english-apa}
  \DeclareLanguageMapping{american}{american-apa}
  %% Bibliography File Path
  \addbibresource{main.bib}
  %% Categories for Specified Bibliography Items
  %%% Category for sources not referenced in-text
  \DeclareBibliographyCategory{consulted}
  \addtocategory{consulted}{noauthor_business_nodate}
  \addtocategory{consulted}{noauthor_college_nodate}


  % Hyperlink Setup
  \hypersetup{
    colorlinks = true,
    urlcolor = blue,
    linkcolor = blue,
    citecolor = blue
  }


  % Page and Text Layout
  \geometry{%
    a4paper,%
    top=1in,%
    bottom=1in,%
    left=1in,%
    right=1in%
  }
  \setlength{\headheight}{15pt}


  % Title Page
  \title{%
    M5D2: Improving YOUR Interpersonal Communication Skills
  }
  \shorttitle{Module 5 Discussion 2}
  \author{Ashton Hellwig}
  \authorsaffiliations{Department of Mathematics, Front Range Community College}
  \course{COM125: Interpersonal Communication}
  \professor{Richard Thomas}
  \duedate{December 12, 2020 23:59:59 MDT}
  \date{\today}
  \lhead{COM125CG1-M5D2}
  \abstract{%
    \textbf{Overview}\\%
    In Module 1 you committed to becoming a better interpersonal communicator by
      focusing on improving two interpersonal communication skills. In each
      subsequent module, we are checking in to see how those efforts are
      proceeding based on your practice sessions. In this module, you will
      report on how your practice has progressed so far.\\%

    You should spend approximately 4 hours on this assignment.%
  }

\begin{document}
  % Title Page
  \maketitle


  \section*{Instructions}
    \begin{enumerate}
      \item How did your interpersonal skills improve over the course of the
        class?
      \item Of the skills you chose to work on, which one improved the most?
        Which one improved the least? Why?
      \item Which skill was the hardest to improve? Which was the easiest?
      \item Which outside influences (e.g., family, work, etc.) interfered with
        your ability to improve your skills? Which influences helped?
      \item At which point in the course do you feel you turned a corner toward
        becoming your ``real you'', as Sam does in the video clip? How did it
        happen? If you do not feel you ever turned a corner, why not?
      \item What skills will you continue to work on improving once the class
        is over? How will you keep yourself motivated to improve those skills
        when you are no longer required to report your efforts to the class?
    \end{enumerate}


  % Initial Post
  \newpage
  \section{Initial Post}
    It has been a great class! Too bad it was so short, I feel like I barely
      got to know anyone here!

    \subsection{How My Skills Have Improved Over the Class}
      I believe my skills have improved over the class due to the fact that I am
        now \textbf{far} more conscience of how my thoughts, feelings, stories,
        and situations which I discuss in public affect other people. I am also
        more aware of how focusing attention on the subject of the conversation
        is far more important than relating the situation being discussed back
        to myself and that silence can be used in a positive way.


    \subsection{Skill Progress}
      I feel like it is not all that possible to be 100\% proficient in
        \textit{every} interpersonal communication skill (or even a single
        one of them), as there are \textbf{always} ways to improve or increase
        one's understanding of another type of person.
      \subsubsection{Skill With Most Improvement}
        The interpersonal communication skill I had been working on with the
          most improvement had to have been empathy, as I will discuss later
          in my post with the subject of being the ``easiest'' (still not easy)
          interpersonal communication skill to work on.
      \subsubsection{Skill With Least Improvement}
        As I will discuss later in this discussion post, I feel my attention to
          the current ``\textit{psychological context}'' is still the skill
          I am most suffering with.


    \subsection{Working on These Skills}
      \subsubsection{The Hardest Skill To Improve}
        The hardest skill to improve by far was the understanding of the current
          ``psychological context'', as stated by one of our course texts
          \parencite{noauthor_communication_2013}.
      \subsubsection{The Easist Skill To Improve}
        The easiest skill for me to improve over this course would have to be
          that of \textit{empathy}. I am now much more affected by the
          situations described to me by others after having experienced many
          low points in my last semester here.
      \subsubsection{Outside Influences Affecting The Use of Certain Skills}
        \paragraph{The Coronavirus Pandemic}
          The obvious outside influence affecting my ability to practice certain
            skills would have to be the Coronavirus COVID-19 Pandemic. While
            I was \textit{always} in online school and in an essential business
            function, COVID-19 did not affect my work or school and in fact
            resulted in me getting \textbf{even more} work hours since February
            of this year.

          The main way in which the Coronavirus COVID-19 Pandemic affected
            the practice of utilizing interpersonal communication sills was
            due to the fact that we all are wearing masks when out in public,
            making the nonverbal communication skills we are all used to
            analyzing for a persons ``true emotions'' non-existent.


    \subsection{Conclusion}
        I believe that I ``turned a corner'' on where I view myself and where
          I would rather be in terms of interpersonal communication skills when
          we discussed the subject of ``\textit{listening}''. This is when I
          noticed I was more of an ``\textit{Action Oriented Listener}''
          or a ``\textit{Time Oriented Listener}''. My ex-girlfriend (turned
          roommate as of recently) expressed how that was one of the main things
          that gave her issues during our five-year long relationship, but was
          \textbf{not} the thing that ended it. Nevertheless, I felt the need
          to work on my interpersonal communication skills as well as show
          her the same things I am working on in order for us both to
          communicate with one another in a more efficient way to improve both
          our current and any future relationships we have.


  % Replies
  %! TEX root=../main.tex

\section{Responses}
  \subsection{Response 1}
    \begin{quotation}
      Placeholder.
    \end{quotation}

    \paragraph{This is a response to FIRST LAST on Post ID 00000000}
      Placeholder.

  % %! TEX root=../main.tex

\subsection{Response 2}
  \begin{quotation}
    I choose an effective delivery method of communication and active listening
      to improve upon because I was not chosen for a promotion at work. I was
      not sure of the reason, but I knew I had some communication issues. I came
      into this class with an open mind and with the intention of figuring out
      what I needed to correct because I was not getting any feedback from
      anyone. Needless to say, I have received my answer.

    [\ldots]

    I will continue to work on all the communication skills I am lacking and
      completing graduate school and receiving a promotion will be my
      motivation. I want my son to communicate well with others in preschool
      and in the future so that he can go further in life than I have.
  \end{quotation}

  \paragraph{This is a response to Diata Hart on Post ID 43614421}
    Placeholder.



  % Bibliography
  %% Works Cited
  \newpage
  \printbibliography[%
    title={References},%
    heading={bibintoc},%
    notcategory={consulted}%
  ]
  %% Works Consulted
  \newpage
  \nocite{*}
  \printbibliography[%
    title={Additional References},%
    heading={bibintoc},%
    category={consulted}%
  ]
\end{document}
