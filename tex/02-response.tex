%! TEX root=../main.tex

\subsection{Response 2}
  \begin{quotation}
    I choose an effective delivery method of communication and active listening
      to improve upon because I was not chosen for a promotion at work. I was
      not sure of the reason, but I knew I had some communication issues. I came
      into this class with an open mind and with the intention of figuring out
      what I needed to correct because I was not getting any feedback from
      anyone. Needless to say, I have received my answer.

    [\ldots]

    I will continue to work on all the communication skills I am lacking and
      completing graduate school and receiving a promotion will be my
      motivation. I want my son to communicate well with others in preschool
      and in the future so that he can go further in life than I have.
  \end{quotation}

  \paragraph{This is a response to Diata Hart on Post ID 43614421}
    Loved your post, Diata! Perhaps we will get to know one another more in a
      future class. I was wondering about the motivation behind your desire
      to improve your interpersonal communication skills: promotion, completing
      graduate degree. Do you think that without the feeling on the inside which
      is what provides the ability to utilize skills such as empathy and a
      stronger understanding of psychological context that many interpersonal
      communication skills can be improved? I would think that if the only goal
      is to make it \textit{seem} like you are listening, this would go against
      the very meaning of what it means to be a great interpersonal communicator
      at its core: the \textit{person} in \textit{interpersonal communication}.
      What are your thoughts?
